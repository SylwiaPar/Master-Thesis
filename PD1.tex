\documentclass[12pt]{report}
\usepackage[a4paper,left=35mm,right=20mm,top=25mm,bottom=25mm]{geometry} 
\usepackage[T1]{fontenc}
\pagestyle{plain}
\parindent 1cm
\def\baselinestretch{1.25}
\usepackage[table,xcdraw]{xcolor}
\usepackage{indentfirst}
\usepackage[font=small]{caption}
\usepackage{pdfpages}
\usepackage{float}
\usepackage[polish]{babel}
\usepackage{polski}
\usepackage[utf8]{inputenc}
\usepackage{tikz}
\usepackage{algpseudocode}
\usepackage{graphics,graphicx} 
\usepackage{longtable}
\usepackage{listings}
\usepackage{pstricks,pst-node,pst-tree,pstricks-add}
\usepackage{listings}
 \lstset{
 inputencoding=utf8x,
 extendedchars=\true
 }
\usepackage{amsmath}
\usepackage{amsthm}

\usepackage{amsfonts}
%\usepackage{amssymb}
\usepackage{pgfplots}
\usepackage{url}

\usepackage{adjustbox}
\usepackage{graphicx}


\theoremstyle{definition}
\newtheorem{tw}{Stwierdzenie}[section]
\newtheorem{prz}[tw]{Przykład}
\newtheorem{df}[tw]{Definicja}
\newtheorem{rem}[tw]{Uwaga}



\begin{document}

\begin{titlepage}
\begin{center}
{\Large\bf{POLITECHNIKA BIAŁOSTOCKA} }
$$ \,$$
{\Large\bf{ WYDZIAŁ INFORMATYKI}}
$$ \,$$
  \bf{ KATEDRA SYSTEMÓW INFORMACYJNYCH I SIECI KOMPUTEROWYCH }
\end{center}
\begin{center}
$$ \,$$
$$ \,$$
{\Large \bf{PRACA DYPLOMOWA MAGISTERSKA}}
$$ \,$$
{\Large \bf{ TEMAT: TESTOWANIE AUTOMATYCZNE APLIKACJI Z WYKORZYSTANIEM NARZĘDZIA JMETER}}\\
\end{center}
$$ \,$$
$$ \,$$
$$ \,$$
$$ \,$$
$$ \,$$
\begin{large}
\begin{flushright}
WYKONAWCA: SYLWIA PARAFIANOWICZ\\
$$ \,$$
PODPIS: ...................................\
\end{flushright}
$$ \,$$
$$ \,$$
\end{large}
\begin{large}
PROMOTOR: DR INŻ. TOMASZ GRZEŚ\\
$$ \,$$
PODPIS: ...................................\
\end{large}
$$ \,$$
\begin{center}
{\large \bf {BIA\L{YSTOK} 2018 ROK} }
\end{center}
\end{titlepage}

\includepdf[]{Karta.pdf}


\setcounter{page}{3}
\textbf{Thesis topic in English:} Automatic application testing using JMeter\\
\begin{center}
	\textbf{SUMMARY}
\end{center}

The purpose of this thesis is to show some interesting solution for a problem with choosing proper automation testing tools, especially for testing web applications. The idea is to use the most popular two: JMeter as a tool in which the test architecture is implemented and Selenium WebDriver for writing scripts in BeanShell Sampler (JMeter's component). The example of usage these tools is presented during testing a simple website on Wordpress.

\newpage

\includepdf[]{osw.pdf}

\setcounter{page}{5}
\tableofcontents

\newpage
\chapter{Wstęp} 


Automatyzacja testowania jest jednym ze sposobów testowania, który przy pomocy odpowiedniego oprogramowania wykonuje testy i porównuje aktualne rezultaty z oczekiwanymi. Ta technika testowania jest bardzo często wykorzystywana w projektach IT, ponieważ dzięki niej można znacznie dokładniej przetestować oprogramowanie niż gdyby to zrobić manualnie, a co za tym idzie zapewnia jego dostarczenie w znacznie krótszym czasie. Ponadto pozwala odciążyć testerów od wykonywania powtarzających się zadań. 


Jednym z kluczowych warunków do sukcesu przy wprowadzaniu automatyzacji procesu testowania jest dobór odpowiednich narzędzi. Celem niniejszej pracy dyplomowej jest ocena przydatności narzędzi, czyli JMetera oraz Selenium, pod kątem testowania aplikacji webowych, które wymagają bardzo wielu przypadków testowych i które są dobrymi kandydatami do automatyzacji. Ponadto, na podstawie przeprowadzonego eksperymentu, zostanie pokazane, że za pomocą tych dwóch narzędzi można w łatwy sposób stworzyć czytelne scenariusze testowe.  Rozwiązanie problemu zostanie zaprezentowane bazując na przykładowej stronie internetowej napisanej z użyciem platformy WordPress. 


Praca została podzielona na pięć rozdziałów. Pierwszy rozdział stanowi wstęp, w którym zaprezentowane zostają cele i motywacje stojące za niniejszą pracą. 


Drugi z nich zawiera definicje dotyczące testowania oprogramowania, w tym też testowania automatycznego, które wprowadzają w tematykę kolejnych rozdziałów. Zawiera też matematyczny aspekt w pracy - wspomniana definicja automatu skończonego.


Kolejny rozdział dotyczy opisu wszystkich programów, jakie będą wykorzystywane w pracy, wraz z krótkim uzasadnieniem ich wyboru w ostatnim podrozdziale. 


Czwarty rozdział stanowi część badawczą - przedstawia środowisko testowe, napisane dla niego przypadki testowe (z ang. \textit{test-cases}) oraz wyniki przeprowadzonych testów automatycznych. Podkreślony zostaje fakt, iż niniejszy eksperyment nie wyczerpuje tematyki pracy, a raczej przedstawia proponowane rozwiązanie na prostym przykładzie strony internetowej.

W ramach ostatniego rozdziału zawarto podsumowanie otrzymanych rezultatów z wyciągniętymi wnioskami i propozycjami rozwinięcia tematyki pracy w przyszłości.


\chapter{Wprowadzająca terminologia} \ \ \

\input r_11


\bigskip \bigskip

\input r_12



\chapter{Oprogramowania wykorzystywane w pracy} \ \ \

\input r_21

\bigskip \bigskip




\chapter{Środowisko testowe i przeprowadzone badania} \ \ \


\input r_31

\bigskip \bigskip

\input r_32



\newpage
\chapter{Podsumowanie} 

Postawioną tezą dla niniejszej pracy było pokazanie rozszerzonych możliwości JMetera jako narzędzia nie tylko do testów wydajnościowych, ale także do pisania automatycznych testów funkcjonalnych, wspomagając się skryptami pisanymi w Selenium WebDriver. Cel został osiągnięty - rezultat eksperymentu opisano dokładnie w poprzednim rozdziale.

Oczywiście, badania te nie wyczerpują tematu automatyzacji testowania za pomocą JMetera wraz z Selenium, lecz do realizacji wyznaczonych w pracy celów są wystarczające. Można rozszerzyć zastosowania tej koncepcji o następujące kwestie:
\begin{itemize}
\item dodanie kroków testowych związanych z zapytaniami bazodanowymi;
\item testowanie wydajności aplikacji webowych wspomagając się skryptami Selenium;
\item wykorzystanie niniejszego rozwiązania do testowania projektów, które wymaga interakcji z większą ilością interfejsów czy systemów, wykorzystując dodatkowe protokoły, takie jak JDBC, HTTP czy SOAP;
\item współdziałanie ze środowiskiem ciągłej integracji (ang. \textit{continuous integration}), takim jak Jenkins czy Travis CI;
\item współpraca z systemami kontroli wersji, takimi jak GIT czy SVN.
\end{itemize}





\newpage

\addcontentsline{toc}{chapter}{Bibliografia} 
\begin{thebibliography}{99}
\bibitem{aut}Anderson B., Witryna internetowa.  \emph{\url{https://medium.com/@briananderson2209/best-automation-testing-tools-for-2018-top-10-reviews}}, stan: 17.06.2018 r.
\bibitem{jmeter}Apache Software Foundation. Witryna internetowa. 
\emph{\url{http://jmeter.apache.org/usermanual/}}, stan: 28.05.2018 r.
\bibitem{comp}Apache Software Foundation. Witryna internetowa. \emph{\url{https://jmeter.apache.org/usermanual/component_reference.html}}, stan: 13.06.2018 r.

\bibitem{myers}Badgett T., Myers G. J., Sandler C., Thomas T. D., \emph{The art of software testing}, 1979
\bibitem{beanshell}BeanShell. Witryna internetowa. \emph{\url{http://www.beanshell.org/intro.html}}, stan: 12.09.2018 r.
\bibitem{webapp}Chaffee A., Witryna internetowa. \emph{\url{http://www.jguru.com/faq/view.jsp?EID=129328}}, stan: 12.09.2018 r.
\bibitem{ieee}\emph{IEEE 289 Standard for Software and System Test Documentation, IEEE Computer Society, 2008}
\bibitem{auto}\emph{IEEE 610.12:1990 Standard Glossary of Software Engineering Terminology, IEEE Computer Society}, 1990
\bibitem{succ} International Software Testing Qualifications Board, \emph{ISTQB - Certified Tester, Expert Level Syllabus - Test Automation - Engineering}, 2014
\bibitem{io} Mantyla M., Rafi D. M., Moses K. R. K., Peterson K., \emph{Benefits and Limitations of Automated Software Testing: Systematic Literature
Review and Practitioner Survey}, 2012.
\bibitem{bs} Niemeyer P., Witryna internetowa. \emph{\url{http://www.beanshell.org/intro.html}}, stan: 17.06.2018 r.
\bibitem{joomla} Oficjalna strona platformy Joomla!, Witryna internetowa, \emph{\url{https://www.joomla.org/about-joomla.html}}, stan: 28.09.2018 r.
\bibitem{lr} Oficjalna strona MicroFocus, \emph{\url{https://software.microfocus.com/en-us/products/loadrunner-load-testing/overview}}, stan: 29.09.2018 r.
\bibitem{stat}Oficjalna strona W3Techs, \emph{\url{https://w3techs.com/technologies/overview/content_management/all}}, stan: 29.09.2018 r.
\bibitem{wpa}Oficjalna strona platformy WordPress. Witryna internetowa. \emph{\url{https://wordpress.org/about/}}, stan: 20.09.2018 r.
\bibitem{roman}Roman A., \emph{Testowanie i jakość oprogramowania}, PWN, Warszawa, 2007
\bibitem{zmitr}Roman A., Zmitrowicz K., \emph{Testowanie oprogramowania w praktyce}, PWN, Warszawa 2017
\bibitem{selenium}SeleniumHQ. Witryna internetowa. \emph{\url{https://www.seleniumhq.org/}}, stan: 12.09.2018 r.
\bibitem{przypadek}SJSI. Witryna internetowa.\emph{\url{http://sjsi.org/slowo/przypadek-testowy/}}, stan: 16.09.2018 r.
\bibitem{smiglin}Smilgin R., \emph{Zawód tester}, PWN, Warszawa, 2016
\bibitem{istqb}SJSI, Witryna internetowa, \emph{\url{http://sjsi.org/slowo/testowanie-bialoskrzynkowe/}}, stan: 16.09.2018 r.



\end{thebibliography}


\addcontentsline{toc}{chapter}{Spis rysunków}
\begingroup
  \let\captionlinebreak\relax
  \listoffigures
\endgroup

\addcontentsline{toc}{chapter}{Spis tabel}
\listoftables

\end{document}